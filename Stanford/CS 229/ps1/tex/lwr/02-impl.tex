\item \subquestionpoints{10} \textbf{Coding problem.}

We will now consider the following dataset:
\begin{center}
	\url{src/lwr/{train,valid,test}.csv}	
\end{center}

Each file contains $\nexp$ examples, one example $(x^{(i)}, y^{(i)})$ per row.
In particular, the $i$-th row contains columns $x^{(i)}_1\in\Re$ and $y^{(i)}\in\{0, 1\}$.

In \texttt{src/lwr/lwr.py}, implement locally weighted linear regression
using the normal equations you derived in Part (a) and using
%
\begin{equation*}
	w^{(i)} = \exp\left(-\frac{\|x^{(i)} - x\|_2^2}{2\tau^2}\right).
\end{equation*}
%
This is a fairly standard choice for weights, where the weight $w^{(i)}$ depends on the particular point $x$ at which we’re trying to evaluate $y$: if $|x^{(i)}-x|$ is small, then $w^{(i)}$ is close to 1; if $|x^{(i)}-x|$ is large, then $w^{(i)}$ is close to 0. Here, $\tau$ is the bandwidth parameter, and it controls how quickly the weight of a training example falls off with distance of its $x^{(i)}$ from the query point $x$.

Train your model on the \texttt{train} split using $\tau = 0.5$, then run your
model on the \texttt{valid} split and report the mean squared error (MSE).
Finally plot your model's predictions on the validation set (plot the
training set with blue `x' markers and the predictions on the
validation set with a red `o' markers). Does the model seem to be under-
or overfitting?
